\documentclass[12pt,letterpaper]{article}
\usepackage[brazil]{babel}
\usepackage[utf8]{inputenc}
\usepackage{graphicx}
\usepackage{times}
\usepackage{url}
\usepackage{algorithm}
\usepackage{algorithmic}
\usepackage[bottom=2cm,top=2cm,left=2cm,right=2cm]{geometry}

\title{Relatório do EP3\\MAC0352 -- Redes de Computadores e Sistemas Distribuídos -- 1/2021}
\author{$<$Aluno$>$ ($<$NUSP$>$)}
\date{}

\begin{document}
\maketitle

\section{Passo 0}

Na definição do protocolo OpenFlow, o que um switch faz toda vez que
ele recebe um pacote que ele nunca recebeu antes?


\section{Passo 2}


Com o acesso à Internet funcionando em sua rede local, instale na VM o
programa \texttt{traceroute} usando \texttt{sudo apt install
traceroute} e escreva abaixo a saída do comando \texttt{sudo
traceroute -I www.inria.fr}. Pela saída do comando, a partir de qual
salto os pacotes alcançaram um roteador na Europa? Como você chegou a
essa conclusão?

\begin{verbatim}
<substitua este conteúdo entre < e > pela saída do traceroute>
<...>
<Obs.: Sempre que for colocar saídas de comandos no relatório, ponha
dentro de um ambiente verbatim como neste exemplo.>
\end{verbatim}

\section{Passo 3 - Parte 1}


Execute o comando \texttt{iperf}, conforme descrito no tutorial, antes
de usar a opção \texttt{--switch user}, 5 vezes.  Escreva abaixo o
valor médio e o intervalo de confiança da taxa retornada (considere
sempre o primeiro valor do vetor retornado).


\section{Passo 3 - Parte 2}


Execute o comando \texttt{iperf}, conforme descrito no tutorial, com a
opção \texttt{--switch user}, 5 vezes. Escreva abaixo o valor médio e
o intervalo de confiança da taxa retornada (considere sempre o
primeiro valor do vetor retornado). O resultado agora corresponde a
quantas vezes menos o da Seção anterior? Qual o motivo dessa
diferença?


\section{Passo 4 - Parte 1}


Execute o comando \texttt{iperf}, conforme descrito no tutorial,
usando o controlador \texttt{of\_tutorial.py} original sem
modificação, 5 vezes. Escreva abaixo o valor médio e o intervalo de
confiança da taxa retornada (considere sempre o primeiro valor do
vetor retornado). O resultado agora corresponde a quantas vezes menos
o da Seção 3? Qual o motivo para essa diferença? Use a saída do
comando \texttt{tcpdump}, deixando claro em quais computadores
virtuais ele foi executado, para justificar a sua resposta.


\section{Passo 4 - Parte 2}


Execute o comando \texttt{iperf}, conforme descrito no tutorial,
usando o seu controlador \texttt{switch.py}, 5 vezes.  Escreva abaixo
o valor médio e o intervalo de confiança da taxa retornada (considere
sempre o primeiro valor do vetor retornado). O resultado agora
corresponde a quantas vezes mais o da Seção anterior?  Qual o motivo
dessa diferença? Use a saída do comando \texttt{tcpdump}, deixando
claro em quais computadores virtuais ele foi executado, para
justificar a sua resposta.


\section{Passo 4 - Parte 3}


Execute o comando \texttt{iperf}, conforme descrito no tutorial,
usando o seu controlador \texttt{switch.py} melhorado, 5 vezes.
Escreva abaixo o valor médio e o intervalo de confiança da taxa
retornada (considere sempre o primeiro valor do vetor retornado). O
resultado agora corresponde a quantas vezes mais o da Seção anterior?
Qual o motivo dessa diferença? Use a saída do comando
\texttt{tcpdump}, deixando claro em quais computadores virtuais ele
foi executado, e saídas do comando \texttt{sudo ovs-ofctl}, com os
devidos parâmetros, para justificar a sua resposta.


\section{Passo 5}


Explique a lógica implementada no seu controlador
\texttt{firewall.py} e mostre saídas de comandos que comprovem que ele
está de fato funcionando (saídas dos comandos \texttt{tcpdump},
\texttt{sudo ovs-ofctl}, \texttt{nc}, \texttt{iperf} e \texttt{telnet}
são recomendadas)


\section{Configuração dos computadores virtual e real usados nas
medições (se foi usado mais de um, especifique qual passo foi feito
com cada um)}

\section{Referências}

\begin{itemize}
   \item
   \item
   % \item (Coloque quantos itens forem necessários)
\end{itemize}

\end{document}
